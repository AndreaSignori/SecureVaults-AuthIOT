\section{Objective}
In the context of IoT device security, mutual authentication between devices and servers represents one of the most critical challenges. Traditional authentication mechanisms based on a single password are vulnerable to attacks such as side- channel and dictionary attacks, making it necessary to adopt more robust and secure approaches. \\
The paper ”Authentication of IoT Device and IoT Server Using Secure Vaults” proposes an authentication protocol based on a multi-key mechanism, in which the shared secret between the server and the IoT device is represented by a ”secure vault,” a collection of keys of equal size. This protocol ensures that, even if one key is compromised, the remaining keys remain secure, and the system is protected from side-channel and dictionary attacks. \\ In this project, we propose to implement a proof of concept of the protocol described in the paper, with the aim of verifying its feasibility and effectiveness in a real context. In particular, we will implement the mutual authentication mechanism based on the secure vault, simulating the inter- action between an IoT device and a server. \\ The implementation will include the generation and management of the secure vault, the challenge-response mechanism for mutual authentication, and the dynamic modification of the secure vault after each communication session.\\ Furthermore, we will analyze the performance of the protocol in terms of security, having it checked with Tamarin-Prover. \\ Through this proof of concept, we intend to validate the effectiveness of the protocol proposed in the paper, providing a practical basis for further research and development

\subsection{Implementation Details}

The implementation will include the following.
\begin{itemize}
    \item Generation and management of the secure vault.
    \item Challenge-response mechanism for mutual authentication.
    \item Dynamic modification of the secure vault after each communication session.
    \item Security performance analysis, verified using Tamarin-Prover.
\end{itemize}

\subsection{Code Availability}

All code created for this project is available in the GitHub repository: \\ 
\href{https://github.com/AndreaSignori/SecureVaults-AuthIOT}{https://github.com/AndreaSignori/SecureVaults-AuthIOT}
